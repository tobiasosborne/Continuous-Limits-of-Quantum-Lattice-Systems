\documentclass[aps,10pt,epsfig,graphics,mathbbm]{article}

\usepackage{amsmath,amsfonts,amssymb,graphics,graphicx,epsfig,color,times,bbm}
\usepackage[sort&compress]{natbib}
\usepackage[usenames,dvipsnames]{xcolor} 
\usepackage{hyperref}
\hypersetup{colorlinks=true,
						linkcolor=MidnightBlue,        % color of internal links (change box color with linkbordercolor)
						citecolor=MidnightBlue,        % color of links to bibliography
						filecolor=magenta,      			 % color of file links
						urlcolor=MidnightBlue          % color of external links}
}


\bibliographystyle{apsrev}


\newcommand{\me}{\mathrm{e}}
\newcommand{\mi}{\mathrm{i}}
\newcommand{\md}{\mathrm{d}}


\newcommand{\T}{\mathbbm{T}}
\newcommand{\cc}{\mathbbm{C}}
\newcommand{\zz}{\mathbbm{Z}}
\newcommand{\nn}{\mathbbm{N}}
\newcommand{\rr}{\mathbbm{R}}
\newcommand{\id}{\mathbbm{1}}
\renewcommand{\vec}[1]{\text{\boldmath$#1$}}

\newtheorem{definition}{Definition}
\newtheorem{theorem}{Theorem}
\newtheorem{lemma}{Lemma}
\newtheorem{corollary}{Corollary}
\newtheorem{property}{Property}
\newtheorem{proposition}{Proposition}
\newtheorem{remark}{Remark}
\newtheorem{example}{Example}
\newtheorem{assumption}{Assumption}

\setlength{\textwidth}{1.4\textwidth}
\setlength{\textheight}{1.25\textheight}
\setlength{\oddsidemargin}{-0.15\oddsidemargin}
%\setlength{\evensidemargin}{-0.3\evensidemargin}
\setlength{\topmargin}{-0.2cm}
\addtolength{\headsep}{-2\headsep}
\bibliographystyle{abbrv}

\newcommand{\proof}{{\em Proof. }}
\newcommand{\qed}{\hfill$\square$\par\vskip24pt}

%\bibliographystyle{unsrt}
\begin{document}



\title{\Large\bf Convergence Rates of Characteristic Functions}

\author{{\large
M. Cramer}\\ \small Institut f\"ur Theoretische Physik, Albert-Einstein
Allee 11, Universit\"at Ulm, Germany}
\date{}
\maketitle

\vspace*{-.6cm}


\maketitle
\begin{abstract}
This is a small note on the rate of convergence in the quantum central limit theorem. So far it only contains the $0$-local and product state case, so in a sense the ``classical'' case.
The more general, ``truly quantum'' case, will be added later. 
\end{abstract}

\section{Setting}

We let $\mathcal{X}$ a collection of lattices sites equipped with a distance $d:\mathcal{X}\times\mathcal{X}\rightarrow \nn$ and consider observables of the form 
\begin{equation}
\hat{X}=\sum_{i\in\mathcal{X}}\hat{X}_i
\end{equation}
with $\hat{X}_i$ hermitian and $R$-local, i.e., it acts only on sites $\{j\in\mathcal{X}\,|\,d(i,j)\le R\}$. For a given state $\hat{\varrho}$ we write $\langle\cdot\rangle=\text{tr}[\hat{\varrho}\;\cdot\,]$ and
\begin{equation}
\begin{split}
\mu=\langle\hat{X}\rangle,\;\;\;\sigma=\langle (\hat{X}-\mu)^2\rangle.
\end{split}
\end{equation}
We will be concerned with the characteristic function $\phi:\rr\rightarrow\cc$,
\begin{equation}
\phi(t)=\langle\me^{\mi t\hat{X}}\rangle,
\end{equation}
and its distance to the corresponding Gaussian characteristic function,
\begin{equation}
\Delta(t)=|\phi(t)-\me^{\mi \mu t-\sigma^2t^2/2}|.
\end{equation}
We will write $|\mathcal{X}|=N$.

\section{Product States and 0-local Observables}
For $0$-local observables and product states we have the following.
\begin{theorem} \label{0-local} Let $R=0$ and $\hat{\varrho}=\otimes_{i\in\mathcal{X}}\hat{\varrho}_i$. If
\begin{equation}
\label{t_condition}
|t|\sum_{i\in\mathcal{X}}\langle(\hat{X}_i-\langle\hat{X}_i\rangle)^2\rangle^{1/2}\langle (\hat{X}_i-\mu_i)^4\rangle^{1/2}
\le \frac{3}{5}\sigma^3
\end{equation}
then
\begin{equation}
\label{the_bound}
\Delta(t/\sigma)\le   \frac{5}{12}\me^{-t^2/4}|t|^3\frac{\sum_{i\in\mathcal{X}}\langle(\hat{X}_i-\langle\hat{X}_i\rangle)^2\rangle^{1/2}\langle (\hat{X}_i-\mu_i)^4\rangle^{1/2}}{\sigma^3}.
\end{equation}
\end{theorem}
{\it Remark 1:} Eq.~\eqref{the_bound} implies
\begin{equation}
\Delta(t/\sigma)\le   \frac{5}{12}\me^{-t^2/4}|t|^3\frac{\sqrt{\sum_{i\in\mathcal{X}}\langle (\hat{X}_i-\mu_i)^4\rangle}}{\sigma^2}.
\end{equation}
{\it Remark 2:} If $\langle(\hat{X}_i-\langle\hat{X}_i\rangle)^2\rangle=s^2$ and $\langle (\hat{X}_i-\mu_i)^4\rangle^{1/2}=\beta$, one recovers the familiar i.i.d. case,
\begin{equation}
\Delta(t/\sigma)\le   \frac{5\beta}{12 s^2}\me^{-t^2/4}|t|^3\frac{1}{\sqrt{N}}\;\;\;\text{if}\;\;\;|t|\le \frac{3s^2}{5\beta}N^{1/2}.
\end{equation}
{\it Remark 3:} If the $\hat{X}_i$ are bounded, $\|\hat{X}_i\|\le x$ then
\begin{equation}
\Delta(t/\sigma)\le \frac{5}{12}\me^{-t^2/4}|t|^3\left(\frac{x^2N}{\sigma^2}\right)^{3/2}\frac{1}{\sqrt{N}}  \;\;\;\text{if}\;\;\;|t|
\le \frac{3}{5}\left(\frac{\sigma^2}{x^2N}\right)^{3/2}\sqrt{N}.
\end{equation}

\noindent
The proof of Theorem~\ref{0-local} is completely analogous to the classical case, see, e.g., pretty much any book on probability theory. We give it for completeness.

\proof
For $i\in\mathcal{X}$ denote
\begin{equation}
\mu_i=\langle\hat{X}_i\rangle,\;\;\;
\sigma_i^2=\langle(\hat{X}_i-\mu_i)^2\rangle.
\end{equation}
As $\hat{\varrho}$ is a product state, we have 
\begin{equation}
\sigma^2=\langle (\hat{X}-\mu)^2\rangle=\sum_{i,j\in\mathcal{X}}\langle(\hat{X}_i-\mu_i)(\hat{X}_j-\mu_j)\rangle
=\sum_{i\in\mathcal{X}}\sigma_i^2
\end{equation}
and
\begin{equation}
\Delta(t)=\me^{-\sigma^2t^2/2}\Bigl|\me^{\sigma^2t^2/2}\prod_{i\in\mathcal{X}}\langle\me^{\mi t(\hat{X}_i-\mu_i)}\rangle-1\Bigr|.
\end{equation}
Now,
\begin{equation}
\begin{split}
\langle\me^{\mi t(\hat{X}_i-\mu_i)}\rangle-1&= -\int_0^t\md x\,\int_0^x\md y\,\bigl\langle(\hat{X}_i-\mu_i)\me^{\mi t(\hat{X}_i-\mu_i)}(\hat{X}_i-\mu_i)\bigr\rangle.
\end{split}
\end{equation}
By the Cauchy--Schwarz inequality
\begin{equation}
\label{CS}
|\langle\hat{A}\hat{B}\rangle|^2=|\text{tr}[\hat{B}\sqrt{\hat{\varrho}}\sqrt{\hat{\varrho}}\hat{A}]|^2\le\text{tr}[\hat{B}\hat{\varrho}\hat{B}^\dagger]\text{tr}[\hat{A}^\dagger\hat{\varrho}\hat{A}]=\langle\hat{A}\hat{A}^\dagger\rangle\langle\hat{B}^\dagger\hat{B}\rangle
\end{equation}
we have
\begin{equation}
\label{second_moment_bound}
\begin{split}
|\langle\me^{\mi t(\hat{X}_i-\mu_i)}\rangle-1|&\le \sigma_i^2t^2/2\le 1/2,
\end{split}
\end{equation}
where the second inequality is implied by Eq.~\eqref{t_condition}.\footnote{$\sigma_i|t|> 1$ implies the absurdity 
$\frac{3}{5}\sum_{i\in\mathcal{X}}\langle (\hat{X}_i-\mu_i)^4\rangle^{1/2}\ge\frac{3}{5}\sigma^2>\sum_{i\in\mathcal{X}}\langle (\hat{X}_i-\mu_i)^4\rangle^{1/2}$.} Hence we may use the principle branch of the logarithm to write
\begin{equation}
\begin{split}
\Delta(t)&=\me^{-\sigma^2t^2/2}\Bigl|\me^{\sigma^2t^2/2}\me^{\sum_{i\in\mathcal{X}}\log\langle\me^{\mi t(\hat{X}_i-\mu_i)}\rangle}-1\Bigr|\\
&=\me^{-\sigma^2t^2/2}\Bigl|\exp\Bigl(\sum_{i\in\mathcal{X}}\bigl[\sigma_i^2 t^2/2+\log\langle\me^{\mi t(\hat{X}_i-\mu_i)}\rangle\bigr]\Bigr)-1\Bigr|\\
&\le \me^{F(t)-\sigma^2t^2/2}F(t), 
\end{split}
\end{equation}
where $F(t)=\sum_iF_i(t)$ with
\begin{equation}
\begin{split}
F_i(t)&=\Bigl|\frac{\sigma_i^2t^2}{2}+\log\langle\me^{\mi t(\hat{X}_i-\mu_i)}\rangle\Bigr|\\
&\le \Bigl|\langle\me^{\mi t(\hat{X}_i-\mu_i)}\rangle-1+\frac{\sigma_i^2t^2}{2}\Bigr|+\Bigl|\langle\me^{\mi t(\hat{X}_i-\mu_i)}\rangle-1-\log\langle\me^{\mi t(\hat{X}_i-\mu_i)}\rangle\Bigr|\\
&=:G_i(t)+H_i(t),
\end{split}
\end{equation}
where we inserted a zero and used the triangle inequality. 
For the first term we have
\begin{equation}
\begin{split}
G_i(t)&=\Bigl|\int_0^t\md x\int_0^x\md y\int_0^y\md z\;\langle  (\hat{X}_i-\mu_i)^2\me^{\mi x(\hat{X}_i-\mu_i)}(\hat{X}_i-\mu_i)\rangle\Bigr|
\end{split}
\end{equation}
such that by Eq.~\eqref{CS}
\begin{equation}
\begin{split}
G_i(t)&\le \int_0^t\md x\int_0^x\md y\int_0^y\md z\;\sqrt{\langle (\hat{X}_i-\mu_i)^4\rangle\langle(\hat{X}_i-\mu_i)^2 \rangle}\\
&=\sigma_i\langle (\hat{X}_i-\mu_i)^4\rangle^{1/2}|t|^3/6.
\end{split}
\end{equation}
For the second term we use the  Mercator series and Eq.~\eqref{second_moment_bound}  to find
\begin{equation}
\begin{split}
H_i(t)
&\le \sum_{n=2}^\infty\frac{1}{n}\bigl|\langle\me^{\mi t(\hat{X}_i-\mu_i)}\rangle-1\bigr|^n
\le \sum_{n=2}^\infty\frac{1}{n}\left(\frac{\sigma_i^2t^2}{2}\right)^n\\
&= \frac{\sigma_i^4t^4}{4}\sum_{n=0}^\infty\frac{1}{n+2}\left(\frac{\sigma_i^2t^2}{2}\right)^n\\
&\le \frac{\sigma_i^4t^4}{4}\sum_{n=0}^\infty\frac{1}{n+2}2^{-n}\\
&\le \frac{\sigma_i^4t^4}{4}.
\end{split}
\end{equation}
Hence,
\begin{equation}
\label{third_line}
\begin{split}
F(t)&\le \frac{|t|^3}{6}\sum_{i\in\mathcal{X}}\sigma_i\langle (\hat{X}_i-\mu_i)^4\rangle^{1/2}+\frac{t^4}{4}\sum_{i\in\mathcal{X}}\sigma_i^4\\
&\le \frac{|t|^3}{6}\sum_{i\in\mathcal{X}}\sigma_i\langle (\hat{X}_i-\mu_i)^4\rangle^{1/2}+\frac{t^4}{4}\sum_{i\in\mathcal{X}}\sigma_i^2\langle (\hat{X}_i-\mu_i)^4\rangle^{1/2}\\
&\le \frac{5}{12}|t|^3\sum_{i\in\mathcal{X}}\sigma_i\langle (\hat{X}_i-\mu_i)^4\rangle^{1/2}\\
&\le \sigma^2\frac{t^2}{4},
\end{split}
\end{equation}
where we used $\sigma_i^2\le \langle (\hat{X}_i-\mu_i)^4\rangle^{1/2}$ (which follows from Eq.~\eqref{CS}) to obtain the second line, Eq.~\eqref{second_moment_bound} to obtain the third line, and Eq.~\eqref{t_condition} to obtain the last line. Thus
\begin{equation}
\begin{split}
\Delta(t)&\le \me^{-\sigma^2t^2/4}F(t)\le  \frac{5}{12}\me^{-\sigma^2t^2/4}|t|^3\sum_{i\in\mathcal{X}}\sigma_i\langle (\hat{X}_i-\mu_i)^4\rangle^{1/2}
\end{split}
\end{equation}
and Eq.~\eqref{the_bound} follows by using the bound in the third line of Eq.~\eqref{third_line}.\qed 


 


\end{document}
